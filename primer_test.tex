% Created 2019-04-07 dom 12:47
\documentclass[12pt]{article}
\usepackage[utf8]{inputenc}
\usepackage[T1]{fontenc}
\usepackage{fixltx2e}
\usepackage{graphicx}
\usepackage{longtable}
\usepackage{float}
\usepackage{wrapfig}
\usepackage{rotating}
\usepackage[normalem]{ulem}
\usepackage{amsmath}
\usepackage{textcomp}
\usepackage{marvosym}
\usepackage{wasysym}
\usepackage{amssymb}
\usepackage{hyperref}
\usepackage[margin=1in]{geometry}
\author{Yábir García Benchakhtir \\yabirgb@correo.ugr.es}
\date{\today}
\title{Apuntes sobre IA: Parte 1}
\hypersetup{
  pdfkeywords={},
  pdfsubject={},
  pdfcreator={Emacs 25.2.2 (Org mode 8.2.10)}}
\begin{document}
\setlength{\parindent}{0cm}
\setlength{\parskip}{3mm}

\maketitle

\section{Agentes}

\subsection{El concepto de Agente}

Un agente es un sistema asociado a una estructura de hardware, dotado
de la capacidad de obtener información de su entorno y actuar en él,
así como responder a los cambios en el mismo. A un agente se le
pide que sea capaz de actuar de forma autónoma y ser flexible para
lograr objetivos planteados.

Cuando hablamos de autonomia nos referimos a que pueda actuar sin la
intervención directa del ser humano, tiene control sobre sus acciones
y su estado interno.

Además los agentes no deberían limitarse a simplemente actuar en
respuesta al entorno en el que se encuentran, también deben ser
capaces de tomar iniciativa y actuar para conseguir objetivos
parciales.

Por úlimo una característica deseable es que sea social para servirse
de la ayuda de otros agentes o de seres humanos para lograr su objetivo.

En resumen, un sistema para ser un agente debe:

\begin{itemize}
\item Percibir su entorno y responder a los cambios que ocurren
\item Ser proactivo, es decir, exhibir comportamientos dirigidos a lograr sus objetivos
\item Ser autonomo, es deicr, actuar sin intervención humana y tener
  control sobre sus acciones y estado.
\end{itemize}

\subsection{Agentes Racionales vs. Agentes Inteligentes}

Decimos que un agente es racional cuando es capaz de actuar
racionalmente. Esto quiere decir que depende de tomar mediciones de su
entorno, tener conocimiento sobre el mismo y elegir entre un conjuntos
de posibles acciones la que mejor resultado tenga para conseguir el
objetivo que se propone.

Un agente inteligente actua de manera similar, toma mediciones de su
entorno usando sensores, tiene un conocimiento del medio pero además
internamenete es capaz de comprender y tiene intencionalidad en sus
acciones.

Es decir, un agente inteligente no actua como una simple función que
relaciona un conjunto de estados a un conjunto de acciones sino que
comprende que es lo que está sucediendo y actua siguiendo su criterio.
Si consideramos el \textit{Test de Turing} como un criterio valido
para evaluar la inteligencia de un agente, podríamos decir que un
agente inteligente es un agente racional capaz de pasar el test de
Turing.

\subsection{Arquitecturas de Agentes}

Si estudiamos la estructura de los agentes de acuerdo a su \textbf{topología}
podemos distinguir tres tipos:

\begin{itemize}
\item Arquitecturas horizontales: Todas las capas que constituyen el
  agente son capaces de percibir estimulos del entorno y realizar
  acciones.
\item Una arquitectura vertical se carazteriza porque una capa es la
  encargada de percibir el entorno y transmite la información de
  manera lineal entre capas hasta una última que es la que se encarga
  de realizar las acciones.
\item Por último en la estructura híbrida tenemos un compartamiento
  similar a la estructura vertical pero es la capa que percibe el
  entorno la que se encarga de realizar las acciones y todas las capas
  en forma de cadena reciben información y transmite información.
\end{itemize}

En función de su nivel nivel de \textbf{abstracción} podemos
clasificar los agentes en:

\begin{itemize}
\item Deliberativos: Cuentan con un modelo del entorno donde se
  encuentran y lo utilizan para planificar sus acciones para conseguir
  su objetivo.
\item Reactivos: Actuan en función del entorno que perciben por sus
  sensores. Operan rápido y eficazmente sin la necesidad de una
  representación interna del entorno. Existen agentes reactivos con
  memoria que almacenan acciones anteriores y pueden llegar a llevar
  una pequeña representación del mundo en el que se mueven.
\item Hibridos: Una mezcla entre reactivo y deliberativo, recomendada
  para muchos problemas, ya que si durante el desarrollo de un plan
  ocurren problemas, se necesitan decisiones reactivas para poder
  esquivar el problema y continuar la ejecución del agente.
\end{itemize}


\section{Características de los Agentes reactivos y deliberativos. Similitudes y diferencias. Arquitecturas.}

Comencemos por entender como funciona cada tipo de agente:

Un agente reactivo tras percibir el entorno a través de sus sensores
procesa la información y construye una representación interna
basandose en lo que ha captado. Entonces escoge una acción y la
transforma en señales para que sea realizada.

Los agentes deliberativos disponen de un modelo del mundo y otro de
los efectos de sus acciones en el mismo. Un agente deliberativo decide
que hacer después de razonar sobre estos modelos con el fin de
conseguir su objetivo.

\textbf{Agentes reactivos:}
\begin{itemize}
\item Su comportamiento se diseña de manera explicita y es necesario
  anticipar todas las situaciones posibles en el medio.
\item Realizan pocos calculos
\item Almacenan todo en memoria
\item Usan arquitecturas horizontales
\end{itemize}

\textbf{Agentes deliberativos}
\begin{itemize}
\item Tienen un conocimiento del mundo
\item Elaboran planes en función de los efectos de sus acciones razonando sobre sus modelos
\item Usan arquitecturas verticales
\end{itemize}

\section{Describir brevemente los métodos de búsqueda no informada.}

Los métodos de búsqueda no informada son estategias de búsqueda en las
que se evalua si el siguiente estado, que a priori no se conoce, es
mejor o peor que el anterior. Se denominan de esta manera porqu no
tenemos conocimiento previo de la longitud o coste de la solución.

\subsection{Búsqueda en anchura}

Algoritmo basado en grafos en el que se expanden los nodos en el orden
que fueron alcanzados. Se caracteriza porque:

\begin{itemize}
\item Siempre encuentra la solución \textit{menos profunda} primero.
\item Claramente es completo ya que recorre todos los posibles estados
\item La solución es óptima supuesto que todas las acciones tengan
  igual costo no negativo.
\end{itemize}

Vemos la frontera como una cola FIFO.

Si notamos por $b$ una cota de los hijos de cada nodo y $d$ la
longitud de la solución menos profunda el coste computaciones es
$\sum_{i=1}^{d}b^i \in O(b^d)$. Cada nodo que se genera se almacena en
memoria luego el coste por almacenar en memoria también es $O(b^d)$.

\includegraphics[width=\textwidth]{anchura}

\subsection{Búsqueda en profundidad}

Se basa en expandir el nodo más profundo de la frontera actual del
árbol de búsqueda.

Vemos la frontera como una cola LIFO. Comunmenete se ven de forma
recursiva.

\begin{itemize}
\item En general la solución encontrada no es optima.
\item La completitud está garantizada solo para la búsqueda en grafos
  y con un espacio finito de estados.
\end{itemize}

Cabe la posibilidad de que se formen bucles si estamos explorando
grafos con profundidad ilimitada. Para solucionar estre problema se
puede usar una constante que limite la longitud del camino, esto se
conoce como \textbf{método de búsqueda en profundidad limitata}.

\includegraphics[width=\textwidth]{profundidad}

\subsection{Búsqueda bidireccional}

Se ejecutan dos búsquedas de manera simultanea. Se ejecuta una
búsqueda desde la raiz hacia adelante y otra desde la solución hacia
atrás (como si estuvieses deshaciendo el camino) parando cuando las
dos se encuentran. Esta búsqueda comprueba si un nodo están en la
frontera del otro arbol antes de expandirlo. Si esto ocurre devuelve
la solución.


\section{Heurísticas}

\subsection{El concepto de heurística}

Una heurística es un criterio, método o principio para decidir que acción
es más apropiada entre un conjunto de acciones para lograr un
objetivo.

Necesitamos que las heurística elegida funcione de la mejor manera
posible por lo que hay que encontrar un equilibrio entre que sean
simples y nos permitan discriminar de la mejor manera posible.

La capacidad heurística es un rasgo característico de nuestra especie,
desde cuyo punto de vista puede describirse como el arte y la ciencia
del descubrimiento y de la invención o de resolver problemas mediante
la creatividad y el pensamiento lateral.

\subsection{Como se construyen las heurísticas}

No se tiene un método científico que permita encontrar
una heurísitca pero una buena opción para construir una es partir de
una simplificación del problema donde se relajan algunas condiciones
del problema.

Tambíen existen tecnicas como\textit{Pattern databases} que son
métodos computables para derivar de manera automática valores
heurísticas de distintas configuraciones las cuales se almacenan en
una base de datos.

\subsection{Uso de las heurísticas en IA}

Cuando en IA necesitamos realizar búsquedasm, recurrimos normalmente
de herurísticas. Una heurística encapsula el conocimiento experto que
se tiene sobre un problema y sirve de guía para el algoritmo de
búsqueda.

Un método heurístico no garantiza que obtengamos la solución óptima,
pero produce resultados satisfactorios. Bajo ciertas circustancias una
heurística puede devolver los resultados óptimos pero requiere
demostrarlo.

Si consideramos los casos que usan mapas algunas heurísticas que se
suelen usar son:

\begin{itemize}
\item En mapas en las que nos movemos en 4 direcciones usamos la
  distancia Manhattan.
\item Si nos podemos mover en las 8 direcciones usamos la distancia Diagonal ($L_\infty$)
\item Si se permite cualquier dirección la distancia euclidea no suele ser buena.
\item En mapas hexagonales en los que nos movemos en 6 direcciones se
  puede usar la distancia Manhattan adaptada.
\end{itemize}

\section{Los métodos de escalada. Caracterización general. Variantes}

Son metodos de búsqueda con información basados en la búsqueda en
profundidad. Son métodos de búsqueda locacl que parten de un nodo
incial y profundizan en la búsqueda seleccionando el mejor hijo,
reiterando mientras el hijo que se encuentra sea mejor que el padre.
Por este motivo no siempre tiene por qué encontrarse una solución, el
método se puede quedar en extremos locales. Si usamos una función para
representar el problema, esto ocurre si la función no es
monótona. Estos metodos no tienen por qué encontrar la solución óptima
pero si encuentran solución al menos encuentran una solución que
óptima localmente.

En la \textbf{Escalada simple} se escoje el primer nodo más cercano,
mientras que en la \textbf{escalada de ascenso pronunciado} todos los
sucesores son comparados y se elige la solución más cercana. Ambas
formas fallan si no existe un nodo cercano, lo cual sucede si hay
máximos locales en el espacio de búsqueda que no son soluciones. La
escalada de ascenso pronunciado es similar a la búsqueda en anchura,
la cual intenta todas las posibles extensiones del camino actual en
vez de sólo una.

El \textbf{descenso coordinado} minimiza a lo largo de una coordenada
para minimizar el valor de una función.

La \textbf{escalada estocástica} escoge aleatoriamente entre los
sucesores con mejor valoración que el estado actual.

La \textbf{escalada de reinicio aleatorio} comienza en diferentes
ocasiones cada vez con una condición inicial aleatoria y se itera
sobre el conjuntos de elementos gurdando la mejor solución obtenida.
Destaca por lo sorprendentemente efectivo que es en muchos casos.

Una manera de evitar que los metodos de escalada se queden estancados
en óptimos locales es usar el algoritmo de enfriamiento simulado que
permite que en el inicio se avance a soluciones peores y la
probabilidad de hacerlo disminuye conforme se avanza en la búsqueda.


\section{Características esenciales de los métodos “primero el mejor”}

Los metodos de ``Primero el mejor'' se caracterizan por ser
estrategias de búsqueda informada en la que se usan algoritmos de
búsqueda en árboles y grafos. Se elige si se explora un nodo o no en
función de la evaluación de una función heurística para ese nodo.

Guarda semejanza con la búsqueda primero en profundidad pero vuelve
atrás cuando no es óptimo contiunuar. Por esto sufre de los mismos
problemas como que no es óptima la búsqueda y es incompleta (puede
profundizar indefinidamente). La complejitad es $O(bm)$ donde m es la
profundidad máxima del espacio de búsqueda pero puede reducirse en
función del problema en particular y la función heurística escogida.

\section{Elementos esenciales del algoritmo A*}

El algoritmo A* es algoritmo de \textbf{Primero el mejor} y una
variante del algoritmo de Djistra.

Se guía por una formula

\[
  F(x) = G(x) + H(x)
\]

donde $G(x)$ es la distancia del nodo inicial hasta el nodo $x$ y
$H(x)$ es una heurística que mide la distancia desde el nodo actual al
objetivo.

Al igual que se hacía en algoritmo de Djistra en cada paso se elige el
nodo más prometedor con al función $F$ y es este el que se expande.
Es una búsqueda en anchura y se puede abandonar una rama y explorar
una nueva.

Podemos decir que los elementos esenciales para un algoritmo A* son:

\begin{itemize}
\item Un grafo de búsqueda
\item Un conjunto de nodos abiertos y nodos cerrados
\item Una función que mida distancias
\item Una función basada en una heurística
\end{itemize}

\section{Elementos esenciales de un algoritmo genético}

Los algoritmos genéticos son muy usados cuando se pretende optimizar
un resultado y cuando se está realizando búsquedas de
soluciones. Aplican estrategias biológicas basadas en la reproducción
sexual, la mutación y en la adaptación.

Se utilizan los siguientes teccnicismos:

\begin{itemize}
\item Cromosomas: Representación de una solución al problema
\item Gen: Atributo concreto del vector de representación de una solución
\item Problación: Conjunto de soluciones que se crean para el problema
\item Adecuación al entorno o fitness: Valor de la función objetivo
\item Selección natural: Operador de selección
\item Mutación: Operador de modificación
\item Cambio generacional: Operador de reemplazamiento
\end{itemize}

El modo en el que funcionan este tipo de algoritmos es el siguiente:

Se genera aleatoriamente una población inicial. Esta población se
constituye de cromosomas que representan las posibles soluciones del
problema.

A cada cromosoma se le aplica una función de \textit{fitness} que
actua como función que nos permite comparar soluciones.

En general se desconoce cual es la solución óptima, así que se
utilizan dos criterios: limitar el número de iteraciones
(generaciones) o detenerlos cuando el cambio que se produzca sea menor
que un humbral. En cada paso se realizan las siguientes acciones:

\begin{enumerate}
\item Selección: Después de conocer la aptitud de cada cromosoma se
  eligen aquellos que se van a utilizar para realizar el cruce (la
  reproducción). Cuanto mayor sea la aptitud, mayor es la probabilidad
  de ser usado para el cruce.

\item Recombinación o cruzamiento: El cruzamiento es el principal
  operador, opera sobre dos cromosomas a la vez y genera dos
  descendientes donde se combinan características de ambos cromosomas
  según algún criterio escogido a priori.

\item Mutación: Al azar se modifica parte del cromosoma para alcanzar
  zonas del espacio de búsqueda que no estaban cubiertas por los
  individuos de la población actual.

\item Reemplazo: Una vez se han aplicado los operadores, se escogen
  los mejores individuos que forman la población de la generación
  siguiente.
\end{enumerate}

\begin{thebibliography}{9}
\bibitem{clase} 
Material proporcionado por el profesor
D. Miguel Delgado Calvo-Florer. Departamento de Computación e inteligencia Artificial, Universidad de Granada
 
\bibitem{dipositivas} 
Foundations of Artificial Intelligence
Problem-Solving Agents, Formulating Problems, Search Strategies
Joschka Boedecker and Wolfram Burgard and Bernhard Nebel, Albert-Ludwigs-Universität Freiburg
\texttt{http://ais.informatik.uni-freiburg.de/teaching/ss17/ki/slides/}
\bibitem{heuristics}
  Heuristics-From Amit’s Thoughts on Pathfinding\\
  \texttt{http://theory.stanford.edu/~amitp/GameProgramming/Heuristics.html}
\bibitem{creatingheuristics}
  \texttt{https://cs.stackexchange.com/questions/19976/creating-admissible-heuristics-
    from-functions}
\end{thebibliography}


\end{document}
