% Created 2019-04-07 dom 12:47
\documentclass[12pt]{article}
\usepackage[utf8]{inputenc}
\usepackage[T1]{fontenc}
\usepackage{fixltx2e}
\usepackage{graphicx}
\usepackage{longtable}
\usepackage{float}
\usepackage{wrapfig}
\usepackage{rotating}
\usepackage[normalem]{ulem}
\usepackage{amsmath}
\usepackage{textcomp}
\usepackage{marvosym}
\usepackage{wasysym}
\usepackage{amssymb}
\usepackage{hyperref}
\usepackage[margin=1in]{geometry}
\author{Yábir García Benchakhtir, Francisco Miguel Castro Macías, \\ Alejandro Miguel Palencia Blanco}
\date{\today}
\title{Apuntes sobre IA: Parte 1}
\hypersetup{
  pdfkeywords={},
  pdfsubject={},
  pdfcreator={Emacs 25.2.2 (Org mode 8.2.10)}}
\begin{document}
\setlength{\parindent}{0cm}
\setlength{\parskip}{3mm}

\maketitle

Este resumen está terminado pero puede tener errores.

\section{Agentes}

\subsection{El concepto de Agente}

Un agente es cualquier cosa capaz de percibir su entorno con la ayuda
de sensores y actuar utilizando actuadores. Entendemos por actuador al
elemento que reacciona a un estímulo mediante una acción.  La
secuencia de percepciones de un agente refleja el historial completo
de lo que el agente ha recibido. Un agente tomará una decisión en un
momento dado dependiendo de la secuencia completa o de una parte de
ella. En términos matemáticos se puede decir que el comportamiento del
agente viene dado por la función del agente que a cada percepción le
asocia una acción.  Un agente que cumpla lo anterior será autónomo (no
necesita al humano) y proactivo (toma sus propias decisiones). Además,
es deseable que sea social para servirse de otros agentes para lograr
su objetivo

\subsection{Agentes Racionales vs. Agentes Inteligentes}

Considerando el test de Turing, diremos que un agente inteligente es
aquel capaz de pasarlo, entendiendo por pasarlo conseguir engañar al
entrevistador que evalua al agente. Es decir, un agente es inteligente
cuando consigue que el entrevistador no sea capaz de decidir si es un
humano o no lo es.

Un agente racional es aquel que intenta para cada situación elegir la
mejor de las opciones disponibles. Requiere de una medida que le diga
lo bueno que es un resultado.

\subsection{Arquitecturas de Agentes}

Si estudiamos la estructura de los agentes de acuerdo a su \textbf{topología}
podemos distinguir tres tipos:

\begin{itemize}
\item Arquitecturas horizontales: Todas las capas que constituyen el
  agente son capaces de percibir estimulos del entorno y realizar
  acciones.
\item Una arquitectura vertical se carazteriza porque una capa es la
  encargada de percibir el entorno y transmite la información de
  manera lineal entre capas hasta una última que es la que se encarga
  de realizar las acciones.
\item Por último en la estructura híbrida tenemos un compartamiento
  similar a la estructura vertical pero es la capa que percibe el
  entorno la que se encarga de realizar las acciones y todas las capas
  en forma de cadena reciben información y transmite información.
\end{itemize}

En función de su nivel nivel de \textbf{abstracción} podemos
clasificar los agentes en:

\begin{itemize}
\item Deliberativos: Cuentan con un modelo del entorno donde se
  encuentran y lo utilizan para planificar sus acciones para conseguir
  su objetivo.
\item Reactivos: Actuan en función del entorno que perciben por sus
  sensores. Operan rápido y eficazmente sin la necesidad de una
  representación interna del entorno. Existen agentes reactivos con
  memoria que almacenan acciones anteriores y pueden llegar a llevar
  una pequeña representación del mundo en el que se mueven.
\item Hibridos: Una mezcla entre reactivo y deliberativo, recomendada
  para muchos problemas, ya que si durante el desarrollo de un plan
  ocurren problemas, se necesitan decisiones reactivas para poder
  esquivar el problema y continuar la ejecución del agente.
\end{itemize}


\section{Características de los Agentes reactivos y deliberativos. Similitudes y diferencias. Arquitecturas.}

Comencemos por entender como funciona cada tipo de agente:

Un agente reactivo tras percibir el entorno a través de sus sensores
procesa la información y construye una representación interna
basandose en lo que ha captado. Entonces escoge una acción y la
transforma en señales para que sea realizada.

Los agentes deliberativos disponen de un modelo del mundo y otro de
los efectos de sus acciones en el mismo. Un agente deliberativo decide
que hacer después de razonar sobre estos modelos con el fin de
conseguir su objetivo.

\textbf{Agentes reactivos:}
\begin{itemize}
\item Su comportamiento se diseña de manera explicita y es necesario
  anticipar todas las situaciones posibles en el medio.
\item Realizan pocos calculos
\item Almacenan todo en memoria
\item Usan arquitecturas horizontales
\end{itemize}

\textbf{Agentes deliberativos}
\begin{itemize}
\item Tienen un conocimiento del mundo
\item Elaboran planes en función de los efectos de sus acciones razonando sobre sus modelos
\item Usan arquitecturas verticales
\end{itemize}


\section{Describir brevemente los métodos de búsqueda no informada.}

Los métodos de búsqueda no informada son estategias de búsqueda en las
que se evalúa si el siguiente estado, que a priori no se conoce, es
mejor o peor que el anterior. Se denominan de esta manera porque no
tenemos conocimiento previo de la longitud o coste de la solución.

\subsection{Búsqueda en anchura}

Algoritmo basado en grafos en el que se expanden los nodos en el orden
que fueron alcanzados. Se caracteriza porque:

\begin{itemize}
\item Siempre encuentra la solución \textit{menos profunda} primero.
\item Claramente es completo ya que recorre todos los posibles estados
\item La solución es óptima supuesto que todas las acciones tengan
  igual costo no negativo.
\end{itemize}

Vemos la frontera como una cola FIFO.

Si notamos por $b$ una cota de los hijos de cada nodo y $d$ la
longitud de la solución menos profunda el coste computacional es
$\sum_{i=1}^{d}b^i \in O(b^d)$. Cada nodo que se genera se almacena en
memoria luego el coste por almacenar en memoria también es $O(b^d)$.

\includegraphics[width=\textwidth]{anchura}

\subsection{Búsqueda de coste uniforme}

Similar a la búsqueda en anchura pero expande el nodo con el costo 
más pequeño del camino que va desde el nodo origen hasta él.

\begin{itemize}
\item Completa y óptima si el costo de cada paso excede una cota positiva $\epsilon$.
\item Si los costes de las acciones son constantes, la búsqueda es idéntica 
  a la búsqueda en anchura.
\end{itemize}

Vemos la frontera como una cola con prioridad en la que la prioridad de cada 
nodo viene dada por su coste (a menor coste, mayor prioridad).

Si $C$ es el coste de la solución óptima, la complejidad en tiempo y espacio 
viene dada por: $O(b^{\frac{C}{\epsilon}})$.

\subsection{Búsqueda en profundidad}

Se basa en expandir el nodo más profundo de la frontera actual del
árbol de búsqueda.

Vemos la frontera como una cola LIFO. Comúnmente se ven de forma
recursiva.

\begin{itemize}
\item En general la solución encontrada no es óptima.
\item La completitud está garantizada solo para la búsqueda en grafos
  y con un espacio finito de estados.
\end{itemize}

Cabe la posibilidad de que se formen bucles si estamos explorando
grafos con profundidad ilimitada. Para solucionar estre problema se
puede usar una constante que limite la longitud del camino, esto se
conoce como \textbf{método de búsqueda en profundidad limitada}.

Siendo $b$ el factor de ramificación en el espacio de estados y 
$m$ la profundidad máxima de cualquier camino, la complejidad 
en tiempo es $O(b^m)$ y en espacio es $O(bm)$.

\includegraphics[width=\textwidth]{profundidad}

\subsection{Búsqueda en profundidad limitada}

Impone un límite de profundidad a la búsqueda primero en profundidad, 
es decir, solo se desarrollarán caminos cuya profundidad no exceda un $m$ fijado.

\subsection{Búsqueda en profundidad iterativa}

Llama a la búsqueda en profundidad limitada aumentando el límite de 
profundidad hasta que se encuentre un objetivo.

\subsection{Búsqueda bidireccional}

Se ejecutan dos búsquedas de manera simultánea. Se ejecuta una
búsqueda desde la raíz hacia adelante y otra desde la solución hacia
atrás (como si estuvieses deshaciendo el camino) parando cuando las
dos se encuentran. Esta búsqueda comprueba si un nodo están en la
frontera del otro árbol antes de expandirlo. Si esto ocurre devuelve
la solución.


\section{Heurísticas}

\subsection{El concepto de heurística}

Una heurística es un criterio, método o principio para decidir que acción
es más apropiada entre un conjunto de acciones para lograr un
objetivo.

Necesitamos que las heurística elegida funcione de la mejor manera
posible por lo que hay que encontrar un equilibrio entre que sean
simples y nos permitan discriminar de la mejor manera posible.

La capacidad heurística es un rasgo característico de nuestra especie,
desde cuyo punto de vista puede describirse como el arte y la ciencia
del descubrimiento y de la invención o de resolver problemas mediante
la creatividad y el pensamiento lateral.

\subsection{Como se construyen las heurísticas}

No se tiene un método científico que permita encontrar
una heurísitca pero una buena opción para construir una es partir de
una simplificación del problema donde se relajan algunas condiciones
del problema.

Tambíen existen tecnicas como\textit{Pattern databases} que son
métodos computables para derivar de manera automática valores
heurísticas de distintas configuraciones las cuales se almacenan en
una base de datos.

\subsection{Uso de las heurísticas en IA}

Cuando en IA necesitamos realizar búsquedasm, recurrimos normalmente
de heurísticas. Una heurística encapsula el conocimiento experto que
se tiene sobre un problema y sirve de guía para el algoritmo de
búsqueda.

Un método heurístico no garantiza que obtengamos la solución óptima,
pero produce resultados satisfactorios. Bajo ciertas circustancias una
heurística puede devolver los resultados óptimos pero requiere
demostrarlo.

Si consideramos los casos que usan mapas algunas heurísticas que se
suelen usar son:

\begin{itemize}
\item En mapas en las que nos movemos en 4 direcciones usamos la
  distancia Manhattan.
\item Si nos podemos mover en las 8 direcciones usamos la distancia Diagonal ($L_\infty$)
\item Si se permite cualquier dirección la distancia euclidea no suele ser buena.
\item En mapas hexagonales en los que nos movemos en 6 direcciones se
  puede usar la distancia Manhattan adaptada.
\end{itemize}


\section{Los métodos de escalada. Caracterización general. Variantes}

Los métodos de escalada se basan en búsqueda local, que consiste en seleccionar 
la solución mejor en el vecindario de una solución inicial, e ir viajando por 
las soluciones del espacio hasta encontrar un óptimo (local o global). Los 
diferentes métodos de búsqueda local se diferencian en la forma en que 
determinan la vecindad de cada nodo. 

Este tipo de métodos se usan en problemas en los que el camino al objetivo no 
es relevante, sino el objetivo en sí, pues durante su ejecución únicamente 
guardan aquellos nodos para los que se va a hallar su vecindad. Esto hace 
que tengan dos ventajas claves con respecto a su eficiencia:
\begin{itemize}
\item Usan muy poca memoria (por lo general una cantidad constante).
\item Pueden encontrar a menudo soluciones razonables en espacios de estados 
  muy grandes o infinitos.
\end{itemize}

Otras características:
\begin{itemize}
\item Informado: Utiliza información del estado para elegir un nodo u otro.
\item No completo: La existencia de máximos locales, crestas y mesetas hacen 
  que este algoritmo no siempre alcance la mejor solución.
\item Evita la exploración de una parte del espacio de estados.
\item El algoritmo mejora considerablemente si la función es monótona.
\end{itemize}

Variantes:
\begin{itemize}
\item Escalada simple: Considera que la vecindad de un nodo es un conjunto de 
  hijos obtenidos secuencialmente hasta que aparece uno mejor que el padre. Así, 
  se parte de un nodo inicial, se obtiene la vecindad de ese nodo, si ningún hijo 
  es mejor que el padre se para, y en caso contrario el primer hijo mejor pasa a 
  ser el nodo en curso y el proceso continúa.
\item Escalada por la máxima pendiente: Considera que la vecindad de un nodo es 
  el conjunto de todos sus hijos obtenidos secuencialmente. Se parte de un nodo 
  inicial, se obtiene su vecindad, se selecciona al hijo mejor que pasa a ser el 
  nodo en curso y el proceso continúa, y en caso de que no se obtenga un hijo 
  mejor que el padre se para.
\item Escalada estocástica: Escoge aleatoriamente entre los sucesores con 
  mejor valoración que el estado actual.
\item Escalada de primera opción: Se generan aleatoriamente sucesores, 
  escogiendo el primero con mejor valoración que el estado actual.
\item Escalada con reinicio aleatorio: Si falla la búsqueda, ésta se repite 
  desde otro estado inicial hasta que encuentre la solución buscada. Si la 
  probabilidad de éxito en la búsqueda es p, entonces el número de reinicios 
  esperado es 1/p. Destaca por lo sorprendentemente efectivo que es en muchos casos.
\item Temple simulado: 
  Está basado en principios de la termodinámica. Su intención es evitar 
  que la búsqueda finalice en un máximo local mediante la elección de algunos 
  movimientos hacia peores soluciones de forma controlada. Se parte de una 
  “temperatura” inicial y en cada iteración se escoge un 
  movimiento aleatorio. Si el movimiento mejora la situación, siempre es 
  aceptado. En caso contrario, la probabilidad de que sea escogido dependerá 
  de la “maldad” de éste (cuanto peor sea, menor probabilidad) y de la 
  temperatura en ese momento (a mayor temperatura, mayor probabilidad). La 
  temperatura baja progresivamente con cada iteración, de modo que los “malos” 
  movimientos serán más probables al comienzo del algoritmo. El temple simulado 
  termina cuando se alcanza una temperatura final deseada.
\end{itemize}


\section{Características esenciales de los métodos “primero el mejor”}

Los metodos de \textit{Primero el mejor} se caracterizan por expandir
el mejor de los nodos encontrados en cada momento bajo la hipotesis de
que esto servirá para acercarse a la solución de manera más rápida.

En su desarrollo se usan dos listas, una de nodos abiertos y otra de
nodos cerrados. En la lista de nodos abiertos mantenemos aquellos
nodos que se han generado pero que todavía no han sido analizados con
la función heurística, es decir, no se han generados sucesores a
partir de ellos. Es una lista que se ordena en función del valor
asignado por la heurística.


En la lista de cerrados manetenemos los nodos que ya se han examinado
y se usa para saber cuando un nodo ha sido visitado con anterioridad.

El procedimiento es el siguiente:

\begin{enumerate}
\item Se crea un grafo de búsqueda G con un nodo inicial \textit{I}
\item Se inicializa la lista de abiertos con I y la de cerrados vacía. Se inicializa también la variable exito a un valor \textit{False}.
\item Mientras Abierto no esté vacío o exito no sea \textit{True}
  \begin{enumerate}
  \item Quitar el primer nodo de abiertos (N) y meterlo en cerrados.
  \item Si N es el estado final entonces exito es \textit{True}
  \item Si no expandir N generando el conjunto S de sucesores de N que
    no son antecesores de N en el grafo.
  \item Generar un nodo en G por cada s de S
  \item Establecer un puntero a N desde aquellos s de S que no
    estuvieran en G y añadirlos a abiertos.
  \item Para cada s de S que estuviera ya en Abiertos o Cerrados decidir si redirigir sus punteros hacia N
  \item Para cada s de S que estuviera en Cerrados decidir si
    rederigir los punteros de los nodos en sus subárboles.
  \item Reordenar Abiertos según f(n)
  \end{enumerate}
\item Si exito entonces Solución es el camino desde I a N a traves de los punteros de G
\item Si no hay exito entonce la solución es un fracaso.
\end{enumerate}


\section{Elementos esenciales del algoritmo A*}

El algoritmo A* es un algoritmo se tipo ``Primero el mejor'' donde la
función de evaluación es la suma de dos componentes:$$f(n)=g(n)+h(n)$$
La función $g(n)$ indica la distancia del mejor camino obtenido hasta
el momento (desde el nodo inicial hasta el nodo $n$), $h(n)$ es la
heurística y expresa la distancia estimada entre el nodo $n$ hasta el
nodo objetivo. Así $f(n)$ expresa el coste más barato estimado de la
solución a través de $n$.

A la hora de llevarlo a cabo se cuenta con una lista de abiertos donde
se encuentran los nodos que se han generado pero aún no se han
expandido, ordenados según la función. También se tiene una lista de
cerrados, donde están los nodos que ya se han expandido. En cada paso
se selecciona el nodo más prometedor de los abiertos y se incluyen en
los cerrados. A continuación se generan todos sus sucesores y se
incluyen en los abiertos. Si el mejor de los nodos abiertos es el nodo
objetivo, se acaba. Si no, se contunúa.

Es una búsqueda en anchura en la cual podemos abandonar temporalmente
una rama que no parace prometedora y continuar con otra.

\subsection{Propiedades}

Si denotamos $h$*$(x)$ al coste del camino óptimo para alcanzar la
solución desde el nodo $x$, el coste del camino óptimo desde el nodo
inicial hasta el final pasando por $x$ es $f$*$(x)=g(x)+h$*$(x)$. Lo
que hacemos es estimar $h$*$(x)$ mediante $h(x)$. Supondremos siempre
que $h(x)\geq 0$.

Una heurística es admirable si cumple que $h(x) \leq h$*$(x)$ para
todo $x$. Una heurística admirable alcanzará la solución óptima. Si
dos heurísticas $h_1(x)$ y $h_2(x)$ cumplen
$h_1(x) \leq h_2(x) \leq h$*$(x)$ para cada nodo $x$ entonces $h_2(x)$
está mejor informada que $h_1(x)$ ya que se acerca más a $h$*$(x)$, y
requerirá menos pasos para alcanzar la solución.

Una heurística es consistente si $h(x) - h(y) \leq c(x,y)$ donde $y$
es el sucesor de $x$ y $c(x,y)$ es el coste del arco que une $x$ con
$y$. En este caso, $F$ es no decreciente y cada vez que se expande un
nodo se ha encontrado el camino óptimo al mismo.

Como limitaciones tenemos que la expansión de un número grande de
nodos puede llevar al algoritmo a agotar la memoria del
computador. Además puede no ser viable encontrar una heurística
admisible y consistente para algún problema.

\section{Elementos esenciales de un algoritmo genético}

Los algoritmos genéticos son muy usados cuando se pretende optimizar
un resultado y cuando se está realizando búsquedas de
soluciones. Aplican estrategias biológicas basadas en la reproducción
sexual, la mutación y en la adaptación.

Se utilizan los siguientes teccnicismos:

\begin{itemize}
\item Cromosomas: Representación de una solución al problema
\item Gen: Atributo concreto del vector de representación de una solución
\item Problación: Conjunto de soluciones que se crean para el problema
\item Adecuación al entorno o fitness: Valor de la función objetivo
\item Selección natural: Operador de selección
\item Mutación: Operador de modificación
\item Cambio generacional: Operador de reemplazamiento
\end{itemize}

El modo en el que funcionan este tipo de algoritmos es el siguiente:

Se genera aleatoriamente una población inicial. Esta población se
constituye de cromosomas que representan las posibles soluciones del
problema.

A cada cromosoma se le aplica una función de \textit{fitness} que
actua como función que nos permite comparar soluciones.

En general se desconoce cual es la solución óptima, así que se
utilizan dos criterios: limitar el número de iteraciones
(generaciones) o detenerlos cuando el cambio que se produzca sea menor
que un humbral. En cada paso se realizan las siguientes acciones:

\begin{enumerate}
\item Selección: Después de conocer la aptitud de cada cromosoma se
  eligen aquellos que se van a utilizar para realizar el cruce (la
  reproducción). Cuanto mayor sea la aptitud, mayor es la probabilidad
  de ser usado para el cruce.

\item Recombinación o cruzamiento: El cruzamiento es el principal
  operador, opera sobre dos cromosomas a la vez y genera dos
  descendientes donde se combinan características de ambos cromosomas
  según algún criterio escogido a priori.

\item Mutación: Al azar se modifica parte del cromosoma para alcanzar
  zonas del espacio de búsqueda que no estaban cubiertas por los
  individuos de la población actual.

\item Reemplazo: Una vez se han aplicado los operadores, se escogen
  los mejores individuos que forman la población de la generación
  siguiente.
\end{enumerate}

\begin{thebibliography}{9}
\bibitem{clase} 
Material proporcionado por el profesor
D. Miguel Delgado Calvo-Florer. Departamento de Computación e inteligencia Artificial, Universidad de Granada
 
\bibitem{dipositivas} 
Foundations of Artificial Intelligence
Problem-Solving Agents, Formulating Problems, Search Strategies
Joschka Boedecker and Wolfram Burgard and Bernhard Nebel, Albert-Ludwigs-Universität Freiburg
\texttt{http://ais.informatik.uni-freiburg.de/teaching/ss17/ki/slides/}
\bibitem{heuristics}
  Heuristics-From Amit’s Thoughts on Pathfinding\\
  \texttt{http://theory.stanford.edu/~amitp/GameProgramming/Heuristics.html}
\bibitem{creatingheuristics}
  \texttt{https://cs.stackexchange.com/questions/19976/creating-admissible-heuristics-
    from-functions}
\end{thebibliography}


\end{document}
